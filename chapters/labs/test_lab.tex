% Feedback in general, why do we test? 
\section{Test Lab}
\link{https://github.com/nymann/TestLab}

In this lab I added tests for two components based on the JavaLab.

\subsection{The reason for automated tests}
\begin{itemize}
        \item Without automated tests, how do you ensure you haven't broken
                anything when implementing new features?
        \item Without automated tests, how can you safely refactor knowing that you have only
                rearranged the code and not altered it's behaviour?
        \item Without automated tests, how can you keep velocity when at the
                same time having to proof to the client that your acceptance
                tests still passes manually?
\end{itemize}
I think it's fitting to quote Kent Beck \cite{kent-beck-xp-history} here since he was one of the founders of
JUnit which is the test framework we are using.

\epigraph{So my experience with programming methodologies started the first day of classes as a computer
science student at the University of Oregon.
The head of the department in the very first class stood up, and gave a
lecture about structured top-down programming, where programmers of the
future will will take a concise precise definition of the problem, and break it
down into code in in such great detail and discipline, that in the in the career
of a programmer of the future you might be able to point to four or five errors
that you made ever. I walked out of that lecture aghast, I'm here, I'm 18
years old I've been programming for probably six, or eight years at that
point, and I just thought oh this cannot be the future, this doesn't sound like
any fun. I'm going to change the way that
people do that...}{\textit{Kent Beck \\ Extreme Programming 20 years later
}}

As someone who usually practices TDD at work (thanks Kent) I couldn't agree
more. Being able to feel confident in your code due to automated tests, makes
programming as a job pleasent.


\subsection{Enemy tests}
\subsubsection{Utilising JUnit setUp}
Instead of each test have to setup a new enemy, we can use JUnit's setup
method.
\inputminted{java}{code/test_lab/enemy/setup.java}
The setUp method will run before each test, that way we get a fresh Enemy each
time, without duplicating code.


\subsubsection{Assert that enemy type is correct}
\inputminted{java}{code/test_lab/enemy/entity_type.java}
Here we are testing the entity type of the enemy, it should always be a ship.
However in general it makes sense to also do a negative test.
It's common to separate the two into multiple functions, the common reason being that you should only
assert one thing.

\subsubsection{Validate "Can Shoot" logic}
% Feedback: if your test does anything else than the name suggest, the name should be different.
%
% Seemingly the reviewer misunderstood this section, so let's make it more clear
The method \texttt{canShoot} does more than it name suggests. It check if the
instance variable \texttt{cooldown} is less or equal to $0$ but it also deducts
the \texttt{cooldown} by the methods \texttt{gameTime} argument. It should
probably have been renamed to \texttt{canShootAndTickCooldown} which would also
make the underlying issue of the method clear.

\inputminted{java}{code/test_lab/enemy/can_shoot.java}

\subsubsection{Validate "Reset Cooldown" logic}
The \texttt{resetCooldown} method sets the cooldown to $0.5f$. In other words an enemy can shoot every $0.5f$ game time.

\inputminted{java}{code/test_lab/enemy/reset_cooldown.java}

\subsection{LifeProcesser tests}
The life processer is a post processing service, that removes entities which
life is 0. It doesn't remove entities that does not have a \texttt{LifePart}.
\inputminted{java}{code/test_lab/life_processer.java}
