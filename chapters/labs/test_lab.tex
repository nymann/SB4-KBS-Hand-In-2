\section{Test Lab}
\link{https://github.com/nymann/TestLab}

In this lab I added tests for two components based on the JavaLab.

\subsection{Enemy tests}
\subsubsection{Utilising JUnit setUp}
Instead of each test have to setup a new enemy, we can use JUnit\'s setup
method.
\inputminted{java}{code/test_lab/enemy/setup.java}
The setUp method will run before each test, that way we get a fresh Enemy each
time, without duplicating code.


\subsubsection{Assert that enemy type is correct}
\inputminted{java}{code/test_lab/enemy/entity_type.java}
Here we are testing the entity type of the enemy, it should always be a ship.
However in general it makes sense to also do a negative test.
It's common to separate the two into multiple functions, the common reason being that you should only
assert one thing.

\subsubsection{Validate "Can Shoot" logic}
The method \texttt{canShoot} does more than it name suggests. It check if the
instance variable \texttt{cooldown} is less or equal to $0$ but it also deducts
the \texttt{cooldown} by the methods \texttt{gameTime} argument.

\inputminted{java}{code/test_lab/enemy/can_shoot.java}

\subsubsection{Validate "Reset Cooldown" logic}
The \texttt{resetCooldown} method sets the cooldown to $0.5f$. In other words an enemy can shoot every $0.5f$ game time.

\inputminted{java}{code/test_lab/enemy/reset_cooldown.java}

\subsection{LifeProcesser tests}
The life processer is a post processing service, that removes entities which
life is 0. It doesn't remove entities that does not have a \texttt{LifePart}.
\inputminted{java}{code/test_lab/life_processer.java}
