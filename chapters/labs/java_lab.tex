% Feedback
% In the JavaLab you should also describe how the SPILocator is used.

\section{Java Lab}
\link{https://github.com/nymann/JavaLab}

See \ref{design-lab} for an analysis of dependencies and possible design
changes.

In this lab java.util.ServiceLoader was used to dynamically register our
service providers. However instead of using the load method from the
ServiceLoader directly inside the Game class, we instead wrap it inside the SPILocator class.

\subsection{SPILocator}
The reason for wrapping ServiceLoader is two-fold, we manage our dependency, making our code more
flexible and thereby easier to change, because if we want to change the
underlying lookup implementation we only have to change code one place (assuming
that our method signature doesn't change), and we also get the benefit of caching,
by storing instances of ServiceLoader to a associative array (HashMap).

Inside SPILocator we have the locateAll method that takes in the parameter service of
a given type and returns a list of the same type.

With that being said let's take a look at the locateAll method.

\begin{minted}{java}
...
ServiceLoader<T> loader = loaderMap.get(service);

if (loader == null) {
        loader = ServiceLoader.load(service);
        loaderMap.put(service, loader);
}
...
\end{minted}

We first check if we already have an instance of the ServiceLoader, if
we don't, instantiate a new service loader for that service type by calling
ServiceLoader.load(service) and put in inside our associate array.
By doing this instead of just using ServiceLoader.load(service) each time, we
speed up future calls to the locateAll method that has a service type we have
already seen once, since retrieving a value from a key in a HashMap is $O(1)$.

Now that we have an instance of a ServiceLoader, we can iterate over it using a
for loop. This is possible because ServiceLoader implements Iterable.
We simply add each service provider found by the loader to an arraylist and
return it.

\subsection{ServiceLoader}
In the previous section we have seen that the actual loading of service
providers is done by java.util.ServiceLoader via it's load method.
It knows about the existence of service providers via provider-configuration
files inside each service providers \texttt{META-INF/services} directory.
From the documentation \cite{oracle-service-loader} we can see that the name of each file should be the fully-qualified binary name of the service type (the interface it implements). The content of the file
should be the fully-qualified binary name of one or more service providers
(implementations) of this service type separated by the newline character.


\subsection{Access}
To get access to the service providers at runtime we call the locateAll method
on the SPILocator class, since it's static we don't need an instance of it.
Inside Game.java in the Core module we have getPluginServices,
getEntityProcessingServices, and getPostEntityProcessingServices.
The each just call locateAll with the matching interface.

\begin{minted}{java}
SPILocator.locateAll(IGamePluginService.class)
\end{minted}

\subsection{Running the game}
\begin{minted}{sh}
# Via make target
make run

# Via maven
mvn install -f AsteroidsEntityFramework/pom.xml
mvn package -f AsteroidsEntityFramework/Core/pom.xml
java -jar bin/JavaLab.jar
\end{minted}

\subsection{Uber jar} \label{java-lab:uberjar}
I encountered a problem in this lab with building a "Uber jar" like I have done
in the other labs, this was due to the newly introduced \texttt{META-INF}
service files in this lab. In the effort of bundling everything together, the
file names conflict for each service provider implementation, resulting in only
one provider implementation for each type. If you were to unzip the jar file
manually, and append an entry to the \texttt{IPostEntityProcessingServicel} file,
everything would work again.

The solution I ended up with to solve this automatically was to switch to
\texttt{maven-shade-plugin} instead of the previously used
\texttt{maven-assembly-plugin}.

In the shade plugin, it is solved by adding a service resource transformer:
\inputminted{xml}{code/maven-shade.xml}

