\section{OSGi Lab}
% you have not explained anything of what was asked.
% Which is:
% - BundleContext, registration and access
% - Declarative services, registration and access-
% 
\link{https://github.com/nymann/OSGiLab}

In the lab we were supposed to use both the bundle context API and delclarative
services. Despite what method you choose, you still have to create a manifest
file for each module that contains the symbolic name of the bundle, which should
be unique since it's used as an identifier.

\begin{minted}{mf}
Bundle-SymbolicName: AsteroidSplitter
\end{minted}

\subsection{Registration}
\subsubsection{Bundle Context}
I implemented the asteroid splitter via the OSGi frameworks' bundle context.
In order to register the service provider we create a class that implements the
BundleActivator interface. From the documentation
\cite{osgi-bundle-activator} we can see that the start method is called when the
bundle is started. We use this to register our asteroid splitting service, the
first parameter \texttt{clazz} is the name of which the service can be found,
it's important that we use the interface name here. By
parsing a new instance of it as the second parameter, and let the third one be
null. The optional third parameter can be used for properties.

In order to make the OSGi framework aware of our Activator, we can write it in
the \texttt{osgi.bnd} file in the module root directory.

\begin{minted}{mf}
Bundle-Activator: 
  dk.sdu.mmmi.cbse.asteroidsplittingsystem.AsteroidSplitterActivator
\end{minted}

\subsubsection{Declarative Service}
One of the advantages of declarative services is the reduction of boilerplate
code \cite{ibm-osgi-demystified}, since we no longer need an activator class.

Instead of the registration being inside a java class, everything is instead
done via XML.

Let's take a look at the player control system, which also contains a reference.

\begin{minted}[fontsize=\small]{xml}
<?xml version="1.0" encoding="UTF-8"?>
<scr:component xmlns:scr="http://www.osgi.org/xmlns/scr/v1.1.0"
  name="dk.sdu.mmmi.cbse.playersystem">
  <implementation class="dk.sdu.mmmi.cbse.playersystem.PlayerControlSystem"/>
  <service>
    <provide interface="dk.sdu.mmmi.cbse.common.services.IEntityProcessingService"/>
  </service>
  <reference
    bind="setBulletService"
    cardinality="0..1"
    interface="dk.sdu.mmmi.cbse.common.bullet.IBulletBehaviour"
    name="BulletService"
    policy="dynamic"
    unbind="removeBulletService"
  />
</scr:component>
\end{minted}

The new thing here is the service tag where we specify which interface thie
service implements (provides).

In order for OSGi to be aware of our xml file, we specify the following in our
\texttt{osgi.bnd} file:

\begin{minted}{mf}
Bundle-SymbolicName: Player
Bundle-ActivationPolicy: lazy
Service-Component: META-INF/processor.xml, META-INF/plugin.xml
\end{minted}

By setting the activation policy to lazy, OSGi will instantiate it just in time
(before it's being called). This means that the game can do the
initial launch without it essentially.

\subsection{Referencing other services}
In order for other modules use the asteroid splitting service in the asteroids module we can
create an xml file \texttt{asteroidprocessor.xml} inside the resources/META-INF directory:

\begin{minted}[fontsize=\small]{xml}
<scr:component
  xmlns:scr="http://www.osgi.org/xmlns/scr/v1.1.0"
  name="dk.sdu.mmmi.cbse.asteroidsystem.AsteroidProcessor">
  <implementation class="dk.sdu.mmmi.cbse.asteroidsystem.AsteroidProcessor"/>
  <service>
    <provide 
    interface="dk.sdu.mmmi.cbse.common.services.IEntityProcessingService"/>
  </service>
  <reference 
    bind="setAsteroidSplitterService" cardinality="0..1"
    interface="dk.sdu.mmmi.cbse.common.asteroid.IAsteroidSplitter"
    name="AsteroidSplitterService"
    policy="dynamic"
    unbind="removeAsteroidSplitterService"
  />
</scr:component>
\end{minted}

Note the reference here, we furthermore specify method names for the bind and
unbind methods. Which we need to implement inside the implementation class
\begin{minted}{java}
...
public void setAsteroidSplitterService(
  IAsteroidSplitter asteroidSplitter
) {
  this.asteroidSplitter = asteroidSplitter;
}

public void removeAsteroidSplitterService(
  IAsteroidSplitter asteroidSplitter
) {
  this.asteroidSplitter = null;
}
...
\end{minted}

By specifying a cardinality of 0 to 1, we are essentially saying that it's
optional, since the application can be started without any asteroid splitter
service bound. If we wanted to require it but only allow one we would simply
specify the cardinality to 1 to 1.

Furthermore when we specify scr:component, we could've also specified more than
just the name, by supplying; activate, deactivate and/or modified, these three
parameters, take the name of a callback function inside the implementation
class. They each might be interesting if we want to be
notified by the OSGi framework when the lifecycle of this component changes
\cite{ibm-osgi-demystified}.

\subsection{Loading entity services and plugins}
In order to do something with all the IGamePlugins, IEntityProcessingServices
and IEntityPostProcessingServices, we need a way to load them inside our Game
class in the Core module.
The method is the same for all three except for the interface type, so I'll show
IEntityProcessingServices as an example.

Inside our META-INF directory in the Core module we specify \texttt{Core.xml}

\begin{minted}{xml}
<?xml version="1.0" encoding="UTF-8"?>
<scr:component xmlns:scr="http://www.osgi.org/xmlns/scr/v1.1.0"
  name="dk.sdu.mmmi.cbse.Game">
  <implementation class="dk.sdu.mmmi.cbse.Game"/>
  ...
  <reference
    bind="addEntityProcessingService"
    cardinality="0..n"
    interface="dk.sdu.mmmi.cbse.common.services.IEntityProcessingService"
    name="IEntityProcessingService"
    policy="dynamic"
    unbind="removeEntityProcessingService"
  />
  ...
</scr:component>
\end{minted}

As we can see, it's the same way as we would reference any other service. We
specify a cardinality of 0 to n, since we want to support 0 or more processing
entity services. Furthermore we define the name of the bind and unbind methods, which we
implement in the Game class.

\begin{minted}{java}
private final List<IEntityProcessingService> entityProcessorList;
...
public Game() {
  entityProcessorList = new CopyOnWriteArrayList();
  ...
}
...

public void addEntityProcessingService(
  IEntityProcessingService eps
) {
  entityProcessorList.add(eps);
}

public void removeEntityProcessingService(
  IEntityProcessingService eps
) {
  entityProcessorList.remove(eps);
}
\end{minted}

We use a copy on write array list for it's thread safety
\cite{java-copy-on-write}, in our case it makes a fresh copy each time we call
add or remove on it.

\subsection{Running the game}
\begin{minted}{sh}
# Via make target
make run

# Via maven
mvn clean install -DskipTests -f Asteroids/pom.xml
mvn install pax:provision -f Asteroids/pom.xml
\end{minted}

\subsection{Color Value fix}
The third party libgdx library we were using had a breaking change in regards
it's function \texttt{Color.valueOf} which I was using to transform a hex color
into a libgdx Color. In my previous labs I was using the format "\#FFFFFF", but
with the version I was using for OSGiLab this no longer worked and I had to
delete the pre-pending pound symbol.
I missed an opportunity to "show-off" my design in regards to this, since I had
made an adapter class for the ShapeRenderer. I could've just updated my setColor
method, instead of changing every caller.

\inputminted{java}{code/color-value-fix.java}


\subsection{Demo}
The dynamic load-unload demo can be seen on
\link{https://github.com/nymann/OSGiLab} or downloaded via:
\link{https://github.com/nymann/OSGiLab/raw/master/demo/osgi-dynamic-load-unload.mp4}
