\section{Spring Lab}
\link{https://github.com/nymann/SpringLab}

\subsection{Theory}
\subsubsection{Registration}
In the Game class I setup an application context with hardcoded xml file names.
\begin{minted}{java}
private final ApplicationContext appContext =
  new ClassPathXmlApplicationContext(
    "Asteroid.xml",
    "AsteroidSplitter.xml", 
    "Enemy.xml",
    "Player.xml",
    "LifeProcesser.xml",
    "CollisionDetection.xml",
    "Bullet.xml"
  );
\end{minted}

The location of each xml file is located in the ressources/META-INF directory
inside the corresponding module for the file (eg. the Asteroid module for
Asteroid.xml).

The content of the beans xml file should just be a bean for each class you want
to make available during run time.

\begin{minted}{xml}
<?xml version="1.0" encoding="UTF-8"?>
<beans
  xmlns="http://www.springframework.org/schema/beans"
  xmlns:xsi="http://www.w3.org/2001/XMLSchema-instance"
  xsi:schemaLocation="http://www.springframework.org/schema/beans 
    http://www.springframework.org/schema/beans/spring-beans.xsd">
    <bean 
      id="AsteroidPlugin"
      class="dk.sdu.mmmi.cbse.asteroidsystem.AsteroidPlugin"
    />
    <bean
      id="AsteroidControlSystem"
      class="dk.sdu.mmmi.cbse.asteroidsystem.AsteroidControlSystem"
    />
</beans>
\end{minted}

\subsubsection{Access}
As in JavaLab we have the 3 methods getPluginServices,
getEntityProcessingServices and getEntityPostProcessingServices which now uses
the spring appContext to return all the beans of the corresponding interface
class type.

\begin{minted}{java}
...
private Collection<? extends IGamePluginService> getPluginServices() {
  return appContext.getBeansOfType(IGamePluginService.class).values();
}
...
\end{minted}


\subsection{Running the game}
\begin{minted}{sh}
# Via make target
make run

# Via maven
mvn install -f Asteroids/pom.xml
mvn package -f Asteroids/Core/pom.xml
java -jar bin/SpringLab.jar
\end{minted}

\subsection{Overview}
In this lab I swapped out the use of \texttt{SPILocator} to use Spring Beans
instead. However, I did not combine the use of NBM and Spring. So my solution is
not that modular, since \texttt{Core} has to know what bean xml files exists.
My first implementation was even less modular, fx to get the Plugins:

\begin{minted}{diff}
- Collection<IGamePluginService> plugins = new ArrayList<>();
- plugins.add(appContext.getBean("AsteroidPlugin", AsteroidPlugin.class));
- plugins.add(appContext.getBean("EnemyPlugin", EnemyPlugin.class));
- plugins.add(appContext.getBean("PlayerPlugin", PlayerPlugin.class));
- plugins.add(appContext.getBean("BulletPlugin", BulletPlugin.class));
- return plugins;
+ return appContext.getBeansOfType(IGamePluginService.class).values();
\end{minted}

A way to re-introduce modularity could also be to supply the xml filepaths as 
parameters via the main method, then we could use constructor injection to supply
the xml files at run time to the Game class.

