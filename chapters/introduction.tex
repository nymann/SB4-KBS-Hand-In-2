\chapter{Introduction}
\section{Readers guide}
All my work related to this course is publicly available at
\link{https://github.com/nymann/SB4-KBS}

\subsection{Obtaining source code}
\inputminted{sh}{code/obtaining-source.sh}

\subsection{Managing build system dependency}
Just like we in this course strive towards abstraction and utilising various
techniques to manage our dependencies, I think it's important to abstract the build and
running process as much as possible, to avoid being tied to an IDE (or even a
specific build system (Maven in this case)). In order to do this \texttt{GNU
Make} is utilised, so the developer can simply invoke \texttt{make run} to run the
application and \texttt{make test} to run the test suite.

The benefit of this is that if at a later point we want to use another build tool like Gradle or Ant,
then we only need to change the Make targets. Furthermore it makes it easier to
make generic continous integration pipelines (which I have implemented via
GitHub Actions for almost all labs), since they can just call the make
targets.
